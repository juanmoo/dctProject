% SIAM Article Template
\documentclass[review,onefignum,onetabnum]{siamart190516}

% Information that is shared between the article and the supplement
% (title and author information, macros, packages, etc.) goes into
% ex_shared.tex. If there is no supplement, this file can be included
% directly.

% SIAM Shared Information Template
% This is information that is shared between the main document and any
% supplement. If no supplement is required, then this information can
% be included directly in the main document.


% Packages and macros go here
\usepackage{lipsum}
\usepackage{amsfonts}
\usepackage{graphicx}
\usepackage{epstopdf}
\usepackage{algorithmic}
\usepackage[caption=false]{subfig}

% Config
\graphicspath{{./figures/}}

\ifpdf
  \DeclareGraphicsExtensions{.eps,.pdf,.png,.jpg}
\else
  \DeclareGraphicsExtensions{.eps}
\fi

% Add a serial/Oxford comma by default.
\newcommand{\creflastconjunction}{, and~}

% Used for creating new theorem and remark environments
\newsiamremark{remark}{Remark}
\newsiamremark{hypothesis}{Hypothesis}
\crefname{hypothesis}{Hypothesis}{Hypotheses}
\newsiamthm{claim}{Claim}


% Title. If the supplement option is on, then "Supplementary Material"
% is automatically inserted before the title.
\title{18.335 Final Project Report}
\author{Juan M Ortiz}


\usepackage{amsopn}
\DeclareMathOperator{\diag}{diag}

% Optional PDF information
\ifpdf
\hypersetup{
  pdftitle={18.335 Final Project Report},
  pdfauthor={Juan M Ortiz}
}
\fi

\begin{document}
\maketitle

% REQUIRED
\begin{abstract}
  \lipsum[1]
\end{abstract}

\section{Introduction}
Along with the numerous benefits that we have come to enjoy in our transition to the Digital Age
have come a plethora of technical challenges. In particular, the rise in the prominence
of the role that data plays in these services has forced us to invent efficient solutions 
to challenges related to storing, retrieving, processing, and transmitting various types of 
data.

One of the most common types of data that we utilize in both computer and mobile environments
is that of images. As such, there has been a lot of research in the field of image compression.
Although the problem of compressing images falls within the general problem of compressing arbitrary
raw binary data, we can greatly improve our performance by designing encoders that take 
advantage of the statistical properties that images posses. Through the use of our domain-specific
knowledge about how images are composed as well as how we perceive images, 
modern compression algorithms are able to remove a great deal of redundancies
so as to create much smaller representations that are able to be decoded into a 
good reconstructions of the originals in accordance to human visual perception.

\subsection{History}


\subsection{Compression Principles}
\lipsum[5-6]

\subsection{Modules of Image Coder}
\lipsum[6-9]

\subsection{Evaluation Metrics}
\lipsum[10-11]

\section{Background}
\subsection{Explanation}
\lipsum[12]

\subsection{Color Specification}
\lipsum[13-14]

\subsection{Discrete Cosine Transform}
\lipsum[15-18]

\subsection{Singular Value Decomposition | Low Rank Approximation}
\lipsum[10-20]

\subsection{Block Algorithms}
\lipsum[21-23]

\subsection{Huffman Coding}
\lipsum[24-25]

\section{Experiments}
\subsection{Image Quality}
Need to test the ability of the algorithms to reconstruct the original image 
over different settings.

Possibly compare the reconstruction accuracy vs the compression

\lipsum[26-30]

\subsection{Performance}
Compare compression vs operations
compare quality vs operations

\lipsum[31-32]

\section{Conclusion}
Nice ass conclusion

\lipsum[33-34]

\bibliographystyle{siamplain}
\bibliography{references}
\end{document}
